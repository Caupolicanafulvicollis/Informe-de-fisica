\documentclass[runningheads]{llncs}
\usepackage[spanish,mexico]{babel}
\usepackage[a4paper, left=3cm, right=2.5cm, top=2cm, bottom=2cm]{geometry}
\usepackage{amsmath}
%\usepackage{amsthm}
%\usepackage{amssymb}
\usepackage{booktabs}
\usepackage[T1]{fontenc}
%\usepackage{mhchem}
\usepackage{graphicx}
\usepackage{array}
\usepackage{booktabs}
\usepackage{colortbl}
\definecolor{gris}{RGB}{242,242,242}
\usepackage[table]{xcolor}
%\usepackage{subcaption}
\usepackage{setspace}
\renewcommand{\baselinestretch}{1.5}
\usepackage[backend=biber,style=ieee]{biblatex}
\addbibresource{referencias.bib}



\begin{document}
    \title{
        Estimación del tiempo de reacción de una persona: Analisis comparativo en el tratamiento del error
    }
    \author{
        Guillermo Enrique Ignacio Vidal Astudillo\orcidID{1,2} 
        \and Mark Antoni Añazco Comicheo\orcidID{1,3} 
        \& Luchiano David Buccioni Azocar\orcidID{1,4}
    }
    \authorrunning{Vidal, Añazco, Buccioni}
    \institute{
        Ingeneria en computación e informática. Departamento de informática, Facultad de ingenería, Universidad Andrés Bello. Chile 
        \and {g.vidalastudillo@uandresbello.edu} 
        \and Enfermero, Universidad San Sebastian. Chile.
        \\{m.aazcocomicheo@uandresbello.edu}
        \and Contador Auditor, Universidad Andrés Bello, Chile. 
        \\{l.buccioniaz@uandresbello.edu}
    }
    
    \maketitle
    
    \begin{abstract}
        El siguiente trabajo de investigación se basa en el análisis de datos en la estimación de la distancia y tiempo de reacción de un reflejo de la motricidad fina en la ejecución pinza índice-pulgar. Los procedimientos utilizados para estimar el tiempo de reacción, 
        incluyeron un análisis estadístico de medición directa e indirecta de esta magnitud física y para la distancia de reacción se desarrolló un análisis estadístico de medida directa. Se emplearon herramientas como cronómetros digitales y reglas para llevar a cabo las mediciones de manera precisa, 
        cuantificando su sensibilidad y error absoluto. Los resultados obtenidos revelaron una alta dispersión de datos en la estimación de la distancia y tiempo de reacción, ya sea una medida indirecta y directa, lo que sugiere la presencia de errores aleatorios en el experimento. 
        La discusión se centra en la importancia de aumentar el número de repeticiones experimentales para mejorar la precisión de las mediciones y reducir los posibles errores sistemáticos. En conclusión, se destaca la necesidad de considerar el tratamiento del error en las investigaciones científicas 
        para obtener resultados más precisos y confiables en la estimación del tiempo de reacción en acciones motrices como la pinza índice-pulgar.
        \keywords{
            sensibilidad \and Teoría del error \and reflejo \and indice \and pulgar \and distancia de reacción \and tiempo de reacción
        }
    \end{abstract}

    \section*{\centering Introducción}
        La ejecución pinza índice-pulgar es uno de los primeros reflejos de la motricidad fina que se presenta en el ser humano y una ejecución motriz, este no solo habla del desarrollo motor, sino que también de la madurez inicial del sistema nervioso y es crucial para ejecutar actividades posteriores como por ejemplo escribir~\cite{gomez2015desarrollo}. 
        Cuando se aplica un estímulo visual a la palma de la mano, los receptores sensoriales cutáneos captan la información y la transmiten a través del sistema nervioso periférico hacia el sistema nervioso central. 
        En el cerebro, esta información es procesada y se generan las señales motoras que activan los músculos responsables de la pinza índice-pulgar. Este proceso ocurre de manera automática y rápida, lo que permite una respuesta precisa y coordinada, pero que también es susceptible de retraso en su ejecución mientras es procesado el estímulo~\cite{guyton1977tratado}. 
        Esta ejecución motriz es única en el ser humano en comparación a los otros en los homínidos que manipulan herramientas como el chimpancé y los grupos de las especies del género Pan~\cite{perezpreferencia}. Para estimar este reflejo con las distancias calculadas y su tiempo debe ser una estimación estadística con un diseño experimental 
        que involucre varias repeticiones de estimación donde los valores de las magnitudes físicas sean los más precisos y exactos porque el error sistemático es significativo debido al bajo tiempo de respuesta del reflejo presente en el ser humano en esta acción de pinza entre el índice y pulgar~\cite{sandoval2021estudio}.
        Es por esta razón que esta investigación se evaluará el tiempo de reacción y su distancia a partir de los lineamientos de teoría del error basados en las medidas de tendencia central bajo la escuela frecuentista de la estadística~\cite{quezada2007potencia}. 
        \linebreak
        En este trabajo de investigación tenemos como objetivo evaluar la precisión en el tiempo y distancia de reacción en una situación experimental de reflejo presente en la ejecución pinza índice-pulgar. Para cumplir con nuestro objetivo debemos dar andamiaje en objetivos específicos, 
        identificar una magnitud física, en este caso el tiempo y distancia de reacción; expresar dichas magnitudes físicas con sus errores asociados; determinar indirectamente el valor de una magnitud física a partir de su derivada parcial en función de otra variable; y por último, 
        estimar el tiempo de reacción en un sujeto experimental como medida directa e indirecta.

    \section*{\centering Modelo Teórico}
    La percepción humana de estímulos está sujeta a sesgos cognitivos que distorsionan nuestra interpretación de la realidad~\cite{firestone2016cognition}. Bajo esta premisa es indispensable pensar en los errores asociados a la misma medición o estimación de los datos, que sin duda deben ser considerados para evitar errores especialmente si de esto depende la implementación de un proyecto.\\
    En este sentido, la teoría del error es un concepto fundamental en la investigación científica y en diversos campos, como la estadística, la ingeniería, la física y la psicología experimental. Se refiere a la noción de que cualquier medición o estimación está sujeta a una cierta cantidad de incertidumbre o imprecisión, conocida como error. 
    Este error puede deberse a una variedad de factores, como la precisión de los instrumentos de medición, la variabilidad inherente en el fenómeno estudiado o los sesgos introducidos por el propio investigador~\cite{Gonzalez}.\\
    El mismo autor afirma que la teoría del error desempeña un papel crucial en la validación y confiabilidad de las mediciones científicas. En este contexto, se distinguen dos tipos principales de errores: los errores sistemáticos y los errores aleatorios. Los primeros, atribuibles a causas identificables y corregibles, impactan de manera consistente todas las mediciones en una dirección específica, como la mala calibración de instrumentos o el uso de fórmulas incorrectas. Por otro lado, los errores aleatorios, derivados de variaciones incontrolables en los factores experimentales, introducen incertidumbre de manera impredecible en las mediciones. Estos conceptos son fundamentales para evaluar tanto la exactitud como la precisión de las mediciones, aspectos esenciales para la interpretación confiable de los resultados científicos.\\
    La comprensión de la teoría del error va más allá de la simple identificación de errores, extendiéndose a la propagación y cuantificación precisa de los mismos en mediciones indirectas. Esto implica el uso de métodos matemáticos para calcular los errores resultantes de operaciones como suma, diferencia, producto y cociente, así como el tratamiento de variables elevadas a potencias. Dichos procedimientos son significativos para estimar la incertidumbre asociada con mediciones indirectas y garantizar la validez de los resultados experimentales. En última instancia, la aplicación rigurosa de la teoría del error no solo mejora la calidad y fiabilidad de la investigación científica, sino que también permite una interpretación más informada y precisa de los fenómenos estudiados~\cite{medina2018propagacion}. \\
    
    \section*{\centering Método experimental}
    
    El procedimiento que consistió para medir el tiempo de reacción de la caída libre se usó una regla de 30 cm, con una sensibilidad de 1 mm que quiere decir 0,001 m, en el cual el agarre por parte del sujeto experimental lo realizó desde la base de la regla para que una vez soltado la regla tomada por otro sujeto se pudiera medir la distancia de reacción sin ninguna distancia extra que no sea cuantificada por el instrumento. \\
    \linebreak
    Para la estimación del tiempo de reacción, otro sujeto midió el tiempo, a través de un cronómetro digital obtenido de la aplicación de un Smartphone con CPU Android, con una sensibilidad de 0,01 s. Así, el sujeto encargado de la regla suelta la regla y en este momento, a viva voz, avisa al otro sujeto para que calcule el tiempo de reacción, en el momento que se suelta el sujeto experimental debe tomar la regla con su índice y pulgar teniendo el brazo estirado, sin apoyarse en un respaldo. \\
    \linebreak
    Una vez soltada la regla, el sujeto experimental debe tomar de la regla con el índice y pulgar y la última medida de la distancia de reacción es la registrada junto con el tiempo estimado. Dicho procedimiento fue repetido 20 veces, por lo tanto, para estimar la medición de la magnitud de la distancia y el tiempo de reacción es una medición estadística. \\
    \linebreak
    Cabe, destacar, que el diseño experimental puede tener error sistemático en su diseño porque hay un margen de error de tiempo no calculado entre la demora que tiene el sujeto en soltar la regla y avisar, en que el sujeto que estima el tiempo de reacción escuche y reaccione al pulsar el botón para que se active el cronómetro, este tiempo son segundos que no se pueden cuantificar. \\
    \linebreak
    Para estimar el valor de la magnitud de la distancia de reacción se consideró como medida estadística del tipo directa donde se usó un instrumento analógico, para estimar la magnitud se usó la ecuación X y para calcular el error absoluto se utilizó el cálculo de la desviación estándar de la población para medida directo (ecuación X) y el cálculo del error instrumental representado en la ecuación X. \\
    \linebreak
    Posteriormente, se desarrolló un análisis comparativo para identificar la precisión de la medición, a través de su error relativo, para estimar la magnitud del tiempo de reacción por medio de un análisis estadístico directo e indirecto. \\
    \linebreak
    \section*{\centering Resultados}
    Una vez tomadas las medidas de la distancia de reacción y el tiempo que demoro bajo 20 repeticiones para evaluar si el método directo o indirecto es el procedimiento más preciso para estimar el tiempo de reacción. A partir de lo anterior se registraron los datos de la distancia (Tabla N$^{\circ}$1) y tiempo de reacción (Tabla N$^{\circ}$2) en sus respectivas unidades de medida del Sistema Internacional (S.I.) Posteriomente se estimo el valor promedio de cada magnitud física.
    \subsection*{Valores medidos para la variable distancia y tiempo} 
    \subsection*{Tabla N°1: Valores medidos de la variable distancia}
    Los valores medidos por la variable distancia están descritos en la Tabla N°1. \\
    \begin{table}[]
        \centering
        \begin{tabular}{crlcrlcrlcr}
            \\ \hline 
        \multicolumn{11}{c}{\textbf{Valores medidos de la distancia}}.\\ \hline 
        \multicolumn{1}{l}{\textbf{N}} & \multicolumn{1}{l}{S(m)} &  & \multicolumn{1}{l}{\textbf{N}} & \multicolumn{1}{l}{S (m)} &  & \multicolumn{1}{l}{\textbf{N}} & \multicolumn{1}{l}{S (m)} &  & \multicolumn{1}{l}{\textbf{N}} & \multicolumn{1}{l}{S (m)} \\ \hline
        \textbf{1} & 0,150 &  & \textbf{6} & 0,120 &  & \textbf{11} & 0,125 &  & \textbf{16} & 0,165 \\
        \textbf{2} & 0,095 &  & \textbf{7} & 0,090 &  & \textbf{12} & 0,120 &  & \textbf{17} & 0,175 \\
        \textbf{3} & 0,065 &  & \textbf{8} & 0,130 &  & \textbf{13} & 0,050 &  & \textbf{18} & 0,120 \\
        \textbf{4} & 0,050 &  & \textbf{9} & 0,165 &  & \textbf{14} & 0,070 &  & \textbf{19} & 0,055 \\
        \textbf{5} & 0,100 &  & \textbf{10} & 0,140 &  & \textbf{15} & 0,115 &  & \textbf{20} & 0,020 \\
        \end{tabular}
        \end{table}
    
        Una vez tomadas las medidas de la distancia de reacción y el tiempo que demoro bajo 20 repeticiones para evaluar si el método directo o indirecto es el procedimiento más preciso para estimar el tiempo de reacción. 
        A partir de los datos obtenidos el valor promedio de la magnitud tiempo es $\bar{S}=0,106m$
    \subsection*{Tabla N°2: Valores medidos de la variable tiempo}
    Los valores medidos por la variable tiempo están descritos en la Tabla N°2.\\
    \begin{table}[]
        \centering
        \begin{tabular}{crlcrlcrlcr}
            \hline 
            \multicolumn{11}{c}{\textbf{Valores medidos del tiempo}} \\ \hline 
            \multicolumn{1}{l}{\textbf{N}} & \multicolumn{1}{l}{t (s)} &  & \multicolumn{1}{l}{\textbf{N}} & \multicolumn{1}{l}{t (s)} &  & \multicolumn{1}{l}{\textbf{N}} & \multicolumn{1}{l}{t (s)} &  & \multicolumn{1}{l}{\textbf{N}} & \multicolumn{1}{l}{t (s)}\\ \hline 
            \textbf{1} & 0,95 &  & \textbf{6} & 0,14 &  & \textbf{11} & 0,07 &  & \textbf{16} & 0,14 \\
            \textbf{2} & 0,80 &  & \textbf{7} & 0,07 &  & \textbf{12} & 0,14 &  & \textbf{17} & 0,07 \\
            \textbf{3} & 0,76 &  & \textbf{8} & 0,08 &  & \textbf{13} & 0,14 &  & \textbf{18} & 0,14 \\
            \textbf{4} & 0,94 &  & \textbf{9} & 0,07 &  & \textbf{14} & 0,07 &  & \textbf{19} & 0,14 \\
            \textbf{5} & 0,41 &  & \textbf{10} & 0,14 &  & \textbf{15} & 0,14 &  & \textbf{20} & 0,40 \\ 
        \end{tabular}
    \end{table}
    A partir de los datos obtenidos el valor promedio de la magnitud tiempo es $\bar{t}=0,29s$


    \subsection*{Estimacion del valor de una magnitud}
    \begin{equation}
        X=\bar{x}\pm \Delta x
    \end{equation}
    Si la medición de la magnitud física de la distancia se llevó a cabo por medio de un método estadístico (N=20), mediante el empleo de un instrumento analógico. Por lo tanto, se calculara su:
    \begin{equation}
        S=\bar{S}\pm \Delta S
    \end{equation}
    Si la medición de la magnitud física del tiempo se llevó a cabo por medio de un método estadístico (N=20), mediante el empleo de un instrumento digital. Por lo tanto, es una medición directa y estadística. \\
    Del mismo modo que se determina el valor de la magnitud de la distancia como su valor promedio y error absoluto, se puede estimar el error absoluto del tiempo a través del método indirecto (ecuación 13) y su valor promedio por medio de la ecuación A.\\
    \begin{equation}
        t=\bar{t}\pm \Delta t
    \end{equation}
    El cálculo del error de la magnitud tiempo se llevará a cabo utilizando tanto el método directo como el indirecto. Inicialmente, así como el método directo para estimar la desviación estándar de la población total según la ecuación 7; posteriormente, se determinará su margen de error utilizando el método indirecto según la ecuación 13.\\
    \subsection*{Error absoluto de una variable}
    \begin{equation}
        \Delta x=2\sigma_{m}+EI
    \end{equation}
    \begin{equation}
        \Delta x=2\frac{\sigma}{\sqrt{N}}+EI
    \end{equation}
\subsection*{Error instrumental}
El error instrumental (E.I.) de la variable distancia se midió con una regla, un instrumento analógico. Por lo tanto, para calcular el E.I. se dividió por dos el valor mínimo que puede estimar la regla, en este caso es de 1 mm, equivalente a 0,001 m cuando se expresa en metros.
    \begin{align*}
        & EI_{S} =\frac{0.001 m}{2} \\
        & EI_{S} = 0.0005 m \tag{A}
    \end{align*}
    El error instrumental (E.I.) de la variable distancia se midió con un cronómetro digital, un instrumento digital. Por lo tanto, para calcular el E.I. se considera solo el valor mínimo que puede estimar el instrumento, en este caso es de 0,01 s.    
    \begin{align*}
        EI_{t} = 0.01 s \tag{B}
    \end{align*}
\subsection*{Desviacion estandar de la poblacion total de una variable}
    \begin{equation}
        \sigma^{2}={\frac{{\displaystyle \sum_{i=1}^{N}{\left(x_{i}-\bar{x}\right)}^{2}}}{N}}
    \end{equation}
    Desviación de la poblacion total para la variable tiempo y distancia
    \begin{equation}
        \sigma_{S}=\sqrt{{\frac{{\displaystyle \sum_{i=1}^{N}{\left(S_{i}-\bar{S}\right)}^{2}}}{N}}}
        \wedge 
        \sigma_{t}=\sqrt{{\frac{{\displaystyle \sum_{i=1}^{N}{\left(t_{i}-\bar{t}\right)}^{2}}}{N}}}
    \end{equation}
    A partir de los calculos demostrados en la Tabla N°1 y N°2 se obtiene la sumatoria de la diferencia de cuadrados y se considera que N=20
    \begin{align*}
        &\sigma_{S}=\sqrt{{\frac{0,839}{20}}}
        &\wedge&
        &\sigma_{t}=\sqrt{{\frac{41,71}{20}}}\\
        &\sigma_{S}=0,2
        &\wedge&
        &\sigma_{t}=1,44 \tag{C}
    \end{align*}
\subsection*{Desviación estandar del promedio o error típico del promedio de una variable} 
    \begin{equation}
        \sigma_{m}=\frac{\sigma}{\sqrt{N}}
    \end{equation}
    \begin{align*}
        &\sigma_{mS}=\frac{\sigma_{S}}{\sqrt{N}}
        &\wedge&
        &\sigma_{mt}=\frac{\sigma_{t}}{\sqrt{N}}\\
        &\sigma_{mS}=\frac{0,2}{\sqrt{20}}
        &\wedge&
        &\sigma_{mt}=\frac{1,44}{\sqrt{20}}\\
        &\sigma_{mS}=\frac{0,2}{4,47}
        &\wedge&
        &\sigma_{mt}=\frac{1,44}{4,47}\\
        &\sigma_{mS}=0,045
        &\wedge&
        &\sigma_{mt}=0,32 \tag{D}\\
    \end{align*}
\subsection*{Error absoluto de la variable tiempo y distancia}
Se toma en cuenta que ambas medidas fueron cuantificadas de forma directa.\\
Se considera la ecuación del error absoluto de una magnitud física, ecuación 5.\\ 
    \begin{align*}
        &\Delta t=2\sigma_{mt}+EI_{t}
        &\wedge&
        &\Delta S=2\sigma_{mS}+EI_{S}
    \end{align*}
    Reemplazando los valores de E.I. obtenidos en el resultado A y B y la desviacion estandar del promedio opbtenidos en D
    \begin{align*}
        &\Delta t=2*0,32+0,01 s
        &\wedge&
        &\Delta S=2*0.045+0,0005 m\\
        &\Delta t=0,65 s
        &\wedge&
        &\Delta S=0,0905 m \tag{E}\\
    \end{align*}
\subsection*{Estimacion del valor de la magintud de la distancia y tiempo}
Como se ha mencionado anteriormente, la variable distancia es una medida directa, por lo tanto, se toma el valor del error absoluto calculado en la resolución E y el valor del promedio obtenido a partir de los datos de la Tabla N°1.\\
Para el tiempo también se considera como una medida directa, por lo tanto, se toma el valor del error absoluto calculado en la resolución E y el valordel promedio obtenido a partir de los datos de la Tabla N°2.\\
    \begin{align*}
        &S=\bar{S}\pm \Delta S
        &\wedge&
        &t=\bar{t}\pm \Delta t\\
        &S=0,106\pm 0,0905 m
        &\wedge&
        &t=0,29\pm 0,65 s \tag{F}\\
    \end{align*}
\subsection*{error relativo y porcentual de una variable}
    \begin{equation}
        E_{r}=\frac{\Delta x}{\bar{x}}
    \end{equation}
    \begin{equation}
        E_{r}=\frac{\Delta x}{\bar{x}}*100\%
    \end{equation}
    A partir de la ecuación anterior se estimó el error relativo y porcentual de cada vairable considerando que ambas son medidas directas. 
    \begin{align*}
        &E_{rs}=\frac{\Delta S}{\bar{S}}
        &\wedge&
        &E_{rt}=\frac{\Delta t}{\bar{t}}\\
        &E_{rs}=\frac{0,095}{0,106}
        &\wedge&
        &E_{rt}=\frac{0,65}{0,29}\\
        &E_{rs}=0,89
        &\wedge&
        &E_{rt}=2,26 \tag{G}\\
        &E_{rs\%}=89\% 
        &\wedge&
        &E_{rt\%}=226\% \tag{H}\\
    \end{align*}
\subsection*{Ecuación de la gravedad}
    \begin{equation}
        S =-\frac{1}{2}gt^{2}
    \end{equation}
    Se considerara el valor de la gravedad como no negativa, por lo tanto se contempla que no es posición sino mas bien altura medida para que la ecuación tenga valores positivos\\
    Se contempla $\vec{g}=9,8 \frac{m}{{s}^{2}}$ y el valor del tiempo y la altura son el valor promedio de cada magnitud física \\
    \begin{align*}
        \bar{S} & =\frac{1}{2}\overrightarrow{g}\bar{t}^{2}\\
        \bar{S} & =\frac{1}{2}\overrightarrow{g}\bar{t}^{2} / *2\\
        2\bar{S}& =\overrightarrow{g}t^{2}/ *\frac{1}{\overrightarrow{g}}\\  
        \frac{2\bar{S}}{\overrightarrow{g}} & =\bar{t}^{2}
    \end{align*}
    \begin{equation}
        \bar{t}=\sqrt{\frac{2\bar{S}}{\overrightarrow{g}}}\\
    \end{equation}    
    A partir de la ecuación anterior (ecuación 12) se puede obtener el promedio de la variable tiempo como una medida indirecta
    Si el promedio de la variable distancia es 0,106 m y la gravedad es $\overrightarrow{g}=9,8 \frac{m}{s^{2}}$
    \begin{align*}
        \bar{t}&=\sqrt{\frac{2*0,106 m}{9,8\frac{m}{s^{2}}}}\\
        \bar{t}&=\sqrt{0,0216 s^{2}}\\
        \bar{t}&=0,15 s\tag{I}\\
    \end{align*}
    \subsection*{Estimación del error en una medida indirecta}
    \begin{equation}
        \Delta F=2\sigma_{m}=2{\sqrt{{\displaystyle\sum_{i=1}^{N}\sigma_{mx}^{2} \frac{\partial F}{\partial x_{i}}^{2}}}}
    \end{equation}
    Para estimar el error absoluto del tiempo como una medida indirecta se debe calcular la derivada parcial y obtener el valor del error absoluto de la distancia estimada en la resolución F.
    \begin{equation}
        \Delta t=2\sigma_{mt}=2{\sqrt{\Delta \bar{S}^{2} \frac{\partial t}{\partial \bar{S}}^{2}}}
    \end{equation}
    Reemplazando los valores obtenidos del error absoluto de la distancia (Resolución F) y el valor de la derivada parcial del tiempo versus distancia (Resolución L), se resuelve:
    \begin{align*}
        \Delta t=2\sigma_{mt}=2{\sqrt{0,095^{2} 0,691^{2}}}\\
        \Delta t=2\sigma_{mt}=2{\sqrt{(0,009025) (0,4774)}}\\
        \Delta t=2\sigma_{mt}=2{\sqrt{0,0656}}\\
        \Delta t=2\sigma_{mt}=2*0,06564\\
        \Delta t=\sigma_{mt}=0,13s\tag{J}\\
    \end{align*}
    \subsection*{Magnitud de la variable tiempo como medida indirecta}
    Para estimar el valor de la magnitud del tiempo como medida indirecta, se estima el error absoluto por medio de la ecuación 14, obteniendose como resultado la resolución J y el valor del promedio se obtiene a partir de la
    ecuación 15, obteniendose como resultado la resolución M.
    \begin{align*}
        &t=\bar{t}\pm \Delta t\\
        &t=0,15 \pm0,13 s
    \end{align*} 
    \subsection{Estimación del error relativo y porcentual de la variable tiempo como medida indirecta}
    A partir de la ecuación 9 y 10 se estimó el error relativo y porcentual de la variable tiempo como medida indirecta. 
    \begin{align*}
        &E_{rt}=\frac{\Delta t}{\bar{t}}\\
        &E_{rt}=\frac{0,13}{0,25}\\
        &E_{rt}=0,52 \tag{K}\\
        &E_{rt\%}=52\% \tag{L}
    \end{align*}
    \subsection*{Calculo de derivada parcial del tiempo en función de la distancia}
    Para obtener la derivada parcial $\frac{\partial t}{\partial S}$ se resuelve por medio de la ecuación 12.
    \begin{align*}
        \bar{t}=\sqrt{\frac{2}{\overrightarrow{g}}}*\sqrt{\bar{S}}\\
    \end{align*}
    Como la gravedad es una constante y la altura es la variable dependiente del tiempo, por lo tanto se deja sola la variable $\bar{S}$ para determinar la derivada parcial\\
    Se considera $\vec{g}=9,8 \frac{m}{{s}^{2}}$
    \begin{align*}
        \bar{t} & =\sqrt{\frac{2}{{9,8 \frac{m}{{s}^{2}}}}}*\sqrt{\bar{S}}\\
        \bar{t} & =\sqrt{0,2}*\sqrt{\bar{S}}\\ 
    \end{align*}
    \begin{equation}
        \bar{t} =0,45*\bar{S}^{\frac{1}{2}}\\
    \end{equation}
    \begin{align*}
        \bar{t} &=0,45\sqrt{\bar{S}}\\
        \bar{t} &=0,45\sqrt{0,106}\\
        \bar{t} &=0,45*0,3256\\
        \bar{t} &=0,147 s \tag{M}\\
    \end{align*}
    Una vez obtenida la ecuación 15 por el despeje de la variable dependiente. Se calcula la derivada parcial en función de la altura. 
    \begin{align*}
        \bar{t}                          & = 0,45*\bar{S}^{\frac{1}{2}} / \frac{\partial}{\partial S} \\
        \frac{\partial t}{\partial S}    & = 0,45*\bar{S}^{\frac{1}{2}-1}\\
        \frac{\partial t}{\partial S}    & = 0,45*\frac{1}{2}*\bar{S}^{-\frac{1}{2}}\\
        \frac{\partial t}{\partial S}    & = 0,225*\frac{1}{\bar{S}^{\frac{1}{2}}}\\
    \end{align*}
    \begin{equation}
        \frac{\partial t}{\partial S}      = \frac{0,225}{\sqrt{\bar{S}}}\\
    \end{equation}
    \begin{align*}
        \frac{\partial t}{\partial S} = \frac{0,225}{\sqrt{0,106}}\\
        \frac{\partial t}{\partial S} = 0,691 \tag{N}\\
    \end{align*}
    \section{Discusión}
    El objetivo presente en esta investigación es evaluar la precisión en el tiempo y distancia de reacción en una situación experimental de reflejo presente en el trabajo de pinza índice-pulgar~\cite{gomez2015desarrollo}. 
    Con el diseño experimental propuesto para la obtención de datos en la determinación del valor de la distancia de reacción se obtuvo una precisión del 89\% 
    obtenido por medio del error porcentual (Resolución H, véase en Resultados), este indicador un alto grado porcentual nos indica que los datos están muy dispersos con su mediana, por lo tanto, el experimento para estimar la distancia de dispersión tiene un alto grado de errores aleatorios~\cite{Ratinoff}. 
    Ante lo anterior, como solución al diseño experimental es aumentar las réplicas, esto quiere decir, aumentar la cantidad de repeticiones del experimento. Aunque el error relativo y porcentual, por medio de la fórmula obtenida (ecuación 9) 
    nos ayuda a estimar la precisión y no la exactitud, a pensar de esto nos cuestionamos que podría existir un posible error sistemático, porque la metodología utilizada en medir la distancia de reacción como se ha señalado anteriormente 
    (Véase en método experimental) el dedo pulgar e índice tenía una distancia relativa con la regla cercana a cero, debido a esta imprecisión podría aumentar el error en estimar en el valor más cercano a la distancia de reacción~\cite{Ratinoff}. 
    En conclusión, se debe aumentar la cantidad de veces que se repitió el experimento, mayor a N=20, para tener un valor más cercano al real, y que la estimación del promedio y el error absoluto no solo sea un evento al azar~\cite{quezada2007potencia}.\\ 
    \linebreak
    Para la estimación de la magnitud del valor del tiempo de reacción, se desarrolló dos procedimientos, uno como medida directa e indirecta. En el primer caso, como medida directa, se calculó el promedio y el error absoluto, 
    tomando la estimación del error como medida directa y estadística desarrollada a partir de la Ecuación 6. A partir del procedimiento anteriormente mencionado, método directo y estadístico, para la estimación del valor de la magnitud del tiempo de reacción se obtuvo un error porcentual del 226\%. 
    Definitivamente, el valor de la magnitud no indica que es un evento azaroso, el experimento para estimar de forma más exacta y precisa el tiempo de respuesta del sujeto experimental, en tomar la regla~\cite{quezada2007potencia}. Este gran porcentaje de error se debe a la imprecisión de los datos al igual que la 
    magnitud de la distancia y se puede deber no solo de un error aleatorio sino más bien otro factor, el error sistemático, porque una posible predicción del posible error en nuestros resultados en metodología 
    (Véase en método experimental) es por el diseño experimental, debido a que el sujeto que tiene la regla sostenida en el aire lista para el agarre del sujeto experimental, en el momento que lo suelta avisa quien toma el tiempo en cronómetro para quien esté a cargo en tomar el tiempo, 
    así activa el cronómetro para estimar el tiempo de reacción en cada situación experimental (replica). A partir de lo mencionado sobre la incertidumbre que genera nuestro diseño experimental podría aumentar o disminuir un tiempo estimado en cada réplica, aumentando el error de la 
    magnitud en ser más inexacta debido al error sistemático debido al diseño experimental~\cite{Ratinoff}.\\
    \linebreak 
    En el segundo caso, para estimar la magnitud física del tiempo de reacción como una medida estadística e indirecta, en la cual para estimar el error se ocupó la ecuación 13 (Véase en Resultados) 
    y para obtener el promedio del tiempo de reacción se usó la ecuación 11, a partir del promedio de la distancia de reacción como variable independiente, la cual la variable dependiente es el tiempo promedio de reacción. 
    A partir de esta estimación de la magnitud física del tiempo de reacción, como variable estadística, indirecta y dependiente de la distancia de reacción, se obtuvo un error porcentual del 52\% (Véase en Resolución L), 
    es el error más bajo que se obtuvo de las tres magnitudes, o sea es el valor más preciso de los tres; sin embargo, a partir de la bibliografía revisada~\cite{quezada2007potencia} este error sigue siendo alto para ser una medida más exacta 
    y precisa al valor real del tiempo de reacción que tiene el sujeto experimental en agarrar la regla. Además, como esta estimación del tiempo de reacción estadística e indirecta es dependiente de la magnitud de la 
    distancia de reacción cuantificada en este informe, por lo tanto, “arrastra” el error obtenido para esta variable porque en este caso no es una variable independiente. Como se ha visto en el caso anterior, 
    se debe aumentar la cantidad de veces que se repite el experimento en la distancia de reacción para que disminuya este error al ser considerada como variable dependiente de la distancia de reacción ~\cite{Ratinoff}~\cite{quezada2007potencia}. 

    \section{Conclusiones}
    Como objetivo general tuvimos que evaluar la precisión en el tiempo y distancia de reacción en una situación experimental de reflejo presente en la ejecución pinza índice-pulgar, para cumplir este objetivo se ha dado andamiaje en 
    objetivos específicos como identificar una magnitud física, en este caso el tiempo y distancia de reacción; expresar dichas magnitudes físicas con sus errores asociados; determinar indirectamente el valor de una magnitud física 
    a partir de su derivada parcial en función de otra variable; y por último, estimar el tiempo de reacción en un sujeto experimental como medida directa e indirecta. A partir de esto, bajo el diseño experimental y análisis de resultados, 
    los objetivos se han cumplido satisfactoriamente para evaluar magnitudes físicas siguiendo las directrices de la teoría del error basada en la escuela frecuentista de la estadística. En este informe hemos evaluado el grado de incertidumbre 
    bajo un pensamiento crítico de nuestro diseño experimental y como proyección de la investigación se debe aumentar la cantidad de réplicas que se hace en este experimento para inferir de forma más exacta y precisa el valor de la magnitud de 
    la distancia y tiempo de reacción. 

\printbibliography
\end{document}
