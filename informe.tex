\documentclass[runningheads]{llncs}
\usepackage[spanish,mexico]{babel}
\usepackage[a4paper, left=3cm, right=2.5cm, top=2cm, bottom=2cm]{geometry}
\usepackage{amsmath}
%\usepackage{amsthm}
%\usepackage{amssymb}
\usepackage{booktabs}
\usepackage[T1]{fontenc}
%\usepackage{mhchem}
\usepackage{graphicx}
\usepackage{array}
\usepackage{booktabs}
\usepackage{colortbl}
\definecolor{gris}{RGB}{242,242,242}
\usepackage[table]{xcolor}
%\usepackage{subcaption}
\usepackage{setspace}
\renewcommand{\baselinestretch}{1.5}
\bibliography{referencias}
\usepackage[backend=biber,citestyle=numeric]{biblatex}





\begin{document}
    \title{
        Estimación del tiempo de reacción de una persona: Analisis comparativo en el tratamiento del error
    }
    \author{
        Guillermo Enrique Ignacio Vidal Astudillo\orcidID{1,2} 
        \and Mark Antoni Añazco Comicheo\orcidID{1,3} 
        \& Luchiano David Buccioni Azocar\orcidID{1,4}
    }
    \authorrunning{Vidal, Añazco, Buccioni}
    \institute{
        Ingeneria en computación e informática. Departamento de informática, Facultad de ingenería, Universidad Andrés Bello. Chile 
        \and Profesor de Biología y Ciencias Naturales, Licenciado en Biología Marina, Pontificia Católica de Chile. Chile.
        \\{g.vidalastudillo@uandresbello.edu} 
        \and Enfermero, Universidad San Sebastian. Chile.
        \\{m.aazcocomicheo@uandresbello.edu}
        \and Contador Auditor, Universidad Andrés Bello, Chile. 
        \\{l.buccioniaz@uandresbello.edu}
    }
    
    \maketitle
    
    \begin{abstract}
        meow meow meow meow meow meow meow meow meow meow meow meow meow 
        guau guau guau guau guau guau guau guau guau guau guau guau guau 
        \keywords{
            gravedad \and caida libre \and Teoría del error
        }
    \end{abstract}

    \section*{\centering Introducción}

    \section*{\centering Modelo Teórico}
    La percepción humana de estímulos está sujeta a sesgos cognitivos que distorsionan nuestra interpretación de la realidad ~\cite{Firestone_article}.\printbibliography. Bajo esta premisa es indispensable pensar en los errores asociados a la misma medición o estimación de los datos, que sin duda deben ser considerados para evitar errores especialmente si de esto depende la implementación de un proyecto. \\
    En este sentido, la teoría del error es un concepto fundamental en la investigación científica y en diversos campos, como la estadística, la ingeniería, la física y la psicología experimental. Se refiere a la noción de que cualquier medición o estimación está sujeta a una cierta cantidad de incertidumbre o imprecisión, conocida como error. 
    Este error puede deberse a una variedad de factores, como la precisión de los instrumentos de medición, la variabilidad inherente en el fenómeno estudiado o los sesgos introducidos por el propio investigador [2].\\
    El mismo autor afirma que la teoría del error desempeña un papel crucial en la validación y confiabilidad de las mediciones científicas. En este contexto, se distinguen dos tipos principales de errores: los errores sistemáticos y los errores aleatorios. Los primeros, atribuibles a causas identificables y corregibles, impactan de manera consistente todas las mediciones en una dirección específica, como la mala calibración de instrumentos o el uso de fórmulas incorrectas. Por otro lado, los errores aleatorios, derivados de variaciones incontrolables en los factores experimentales, introducen incertidumbre de manera impredecible en las mediciones. Estos conceptos son fundamentales para evaluar tanto la exactitud como la precisión de las mediciones, aspectos esenciales para la interpretación confiable de los resultados científicos.\\
    La comprensión de la teoría del error va más allá de la simple identificación de errores, extendiéndose a la propagación y cuantificación precisa de los mismos en mediciones indirectas. Esto implica el uso de métodos matemáticos para calcular los errores resultantes de operaciones como suma, diferencia, producto y cociente, así como el tratamiento de variables elevadas a potencias. Dichos procedimientos son significativos para estimar la incertidumbre asociada con mediciones indirectas y garantizar la validez de los resultados experimentales. En última instancia, la aplicación rigurosa de la teoría del error no solo mejora la calidad y fiabilidad de la investigación científica, sino que también permite una interpretación más informada y precisa de los fenómenos estudiados [3].\\
    
    \section*{\centering Método experimental y resultados}
    
    El procedimiento que consistió para medir el tiempo de reacción de la caída libre se usó una regla de 30 cm, con una sensibilidad de 1 mm que quiere decir 0,001 m, en el cual el agarre por parte del sujeto experimental lo realizó desde la base de la regla para que una vez soltado la regla tomada por otro sujeto se pudiera medir la distancia de reacción sin ninguna distancia extra que no sea cuantificada por el instrumento. \\
    \linebreak
    Para la estimación del tiempo de reacción, otro sujeto midió el tiempo, a través de un cronómetro digital obtenido de la aplicación de un Smartphone con CPU Android, con una sensibilidad de 0,01 s. Así, el sujeto encargado de la regla suelta la regla y en este momento, a viva voz, avisa al otro sujeto para que calcule el tiempo de reacción, en el momento que se suelta el sujeto experimental debe tomar la regla con su índice y pulgar teniendo el brazo estirado, sin apoyarse en un respaldo. \\
    \linebreak
    Una vez soltada la regla, el sujeto experimental debe tomar de la regla con el índice y pulgar y la última medida de la distancia de reacción es la registrada junto con el tiempo estimado. Dicho procedimiento fue repetido 20 veces, por lo tanto, para estimar la medición de la magnitud de la distancia y el tiempo de reacción es una medición estadística. \\
    \linebreak
    Cabe, destacar, que el diseño experimental puede tener error sistemático en su diseño porque hay un margen de error de tiempo no calculado entre la demora que tiene el sujeto en soltar la regla y avisar, en que el sujeto que estima el tiempo de reacción escuche y reaccione al pulsar el botón para que se active el cronómetro, este tiempo son segundos que no se pueden cuantificar. \\
    \linebreak
    Para estimar el valor de la magnitud de la distancia de reacción se consideró como medida estadística del tipo directa donde se usó un instrumento analógico, para estimar la magnitud se usó la ecuación X y para calcular el error absoluto se utilizó el cálculo de la desviación estándar de la población para medida directo (ecuación X) y el cálculo del error instrumental representado en la ecuación X. \\
    \linebreak
    Posteriormente, se desarrolló un análisis comparativo para identificar la precisión de la medición, a través de su error relativo, para estimar la magnitud del tiempo de reacción por medio de un análisis estadístico directo e indirecto. \\
    \linebreak
    \section*{\centering Resultados}
    Una vez tomadas las medidas de la distancia de reacción y el tiempo que demoro bajo 20 repeticiones para evaluar si el método directo o indirecto es el procedimiento más preciso para estimar el tiempo de reacción.A partir de lo anterior se registraron los datos de la distancia (Tabla N°1) y tiempo de reacción (Tabla N°2) en sus respectivas unidades de medida del Sistema Internacional (S.I.). Posteriomente se estimo el valor promedio de cada magnitud física. \\
    \subsection*{Valores medidos para la variable distancia y tiempo} 
    \subsection*{Tabla N°1: Valores medidos de la variable distancia}
    Los valores medidos por la variable distancia están descritos en la Tabla N°1. \\
    \begin{table}[]
        \centering
        \begin{tabular}{crlcrlcrlcr}
        \hline
        \multicolumn{11}{c}{\textbf{Valores medidos de la distancia}} \\ \hline
        \multicolumn{1}{l}{\textbf{N}} & \multicolumn{1}{l}{S(m)} &  & \multicolumn{1}{l}{\textbf{N}} & \multicolumn{1}{l}{S(m)} &  & \multicolumn{1}{l}{\textbf{N}} & \multicolumn{1}{l}{S(m)} &  & \multicolumn{1}{l}{\textbf{N}} & \multicolumn{1}{l}{S(m)} \\ \hline
        \textbf{1} & 0,150 &  & \textbf{6} & 0,120 &  & \textbf{11} & 0,125 &  & \textbf{16} & 0,165 \\
        \textbf{2} & 0,095 &  & \textbf{7} & 0,090 &  & \textbf{12} & 0,120 &  & \textbf{17} & 0,175 \\
        \textbf{3} & 0,065 &  & \textbf{8} & 0,130 &  & \textbf{13} & 0,050 &  & \textbf{18} & 0,120 \\
        \textbf{4} & 0,050 &  & \textbf{9} & 0,165 &  & \textbf{14} & 0,070 &  & \textbf{19} & 0,055 \\
        \textbf{5} & 0,100 &  & \textbf{10} & 0,140 &  & \textbf{15} & 0,115 &  & \textbf{20} & 0,020 \\ \hline
        \end{tabular}
        \end{table}
    
        Una vez tomadas las medidas de la distancia de reacción y el tiempo que demoro bajo 20 repeticiones para evaluar si el método directo o indirecto es el procedimiento más preciso para estimar el tiempo de reacción. 
        A partir de los datos obtenidos el valor promedio de la magnitud tiempo es $\bar{S}=0,106m$
    \subsection*{Tabla N°2: Valores medidos de la variable tiempo}
    Los valores medidos por la variable tiempo están descritos en la Tabla N°2.\\
    \begin{table}[]
        \centering
        \begin{tabular}{crlcrlcrlcr}
            \hline
            \multicolumn{11}{c}{\textbf{Valores medidos del tiempo}} \\ \hline
            \multicolumn{1}{l}{\textbf{N}} & \multicolumn{1}{l}{t (s)} &  & \multicolumn{1}{l}{\textbf{N}} & \multicolumn{1}{l}{t (s)} &  & \multicolumn{1}{l}{\textbf{N}} & \multicolumn{1}{l}{t (s)} &  & \multicolumn{1}{l}{\textbf{N}} & \multicolumn{1}{l}{t (s)} \\ \hline
            \textbf{1} & 0,95 &  & \textbf{6} & 0,14 &  & \textbf{11} & 0,07 &  & \textbf{16} & 0,14 \\
            \textbf{2} & 0,80 &  & \textbf{7} & 0,07 &  & \textbf{12} & 0,14 &  & \textbf{17} & 0,07 \\
            \textbf{3} & 0,76 &  & \textbf{8} & 0,08 &  & \textbf{13} & 0,14 &  & \textbf{18} & 0,14 \\
            \textbf{4} & 0,94 &  & \textbf{9} & 0,07 &  & \textbf{14} & 0,07 &  & \textbf{19} & 0,14 \\
            \textbf{5} & 0,41 &  & \textbf{10} & 0,14 &  & \textbf{15} & 0,14 &  & \textbf{20} & 0,40 \\ \hline
        \end{tabular}
    \end{table}
    A partir de los datos obtenidos el valor promedio de la magnitud tiempo es $\bar{t}=0,29s$


    \subsection{Estimacion del valor de una magnitud}
    \begin{equation}
        X=\bar{x}\pm \Delta x
    \end{equation}
    Si la medición de la magnitud física de la distancia se llevó a cabo por medio de un método estadístico (N=20), mediante el empleo de un instrumento analógico. Por lo tanto, se calculara su : 
    \begin{equation}
        S=\bar{S}\pm \Delta S
    \end{equation}
    Si la medición de la magnitud física del tiempo se llevó a cabo por medio de un método estadístico (N=20), mediante el empleo de un instrumento digital. Por lo tanto, es una medición directa y estadística. \\
    Del mismo modo que se determina el valor de la magnitud de la distancia como su valor promedio y error absoluto, se puede estimar el error absoluto del tiempo a través del método indirecto (ecuación 13) y su valor promedio por medio de la ecuación A.\\
    \begin{equation}
        t=\bar{t}\pm \Delta t
    \end{equation}
    El cálculo del error de la magnitud tiempo se llevará a cabo utilizando tanto el método directo como el indirecto. Inicialmente, así como el método directo para estimar la desviación estándar de la población total según la ecuación 7; posteriormente, se determinará su margen de error utilizando el método indirecto según la ecuación 13.\\
    \subsection{Error absoluto de una variable}
    \begin{equation}
        \Delta x=2\sigma_{m}+EI
    \end{equation}
    \begin{equation}
        \Delta x=2\frac{\sigma}{\sqrt{N}}+EI
    \end{equation}
\subsection{Error instrumental}
El error instrumental (E.I.) de la variable distancia se midió con una regla, un instrumento analógico. Por lo tanto, para calcular el E.I. se dividió por dos el valor mínimo que puede estimar la regla, en este caso es de 1 mm, equivalente a 0,001 m cuando se expresa en metros.
    \begin{align*}
        & EI_{S} =\frac{0.001 m}{2} \\
        & EI_{S} = 0.0005 m \tag{A}
    \end{align*}
    El error instrumental (E.I.) de la variable distancia se midió con un cronómetro digital, un instrumento digital. Por lo tanto, para calcular el E.I. se considera solo el valor mínimo que puede estimar el instrumento, en este caso es de 0,01 s.    
    \begin{align*}
        EI_{t} = 0.01 s \tag{B}
    \end{align*}
\subsection{Desviacion estandar de la poblacion total de una variable}
    \begin{equation}
        \sigma^{2}={\frac{{\displaystyle \sum_{i=1}^{N}\left(x_{i}-\bar{x}\right)^{2}}}{N}}
    \end{equation}
    Desviación de la poblacion total para la variable tiempo y distancia
    \begin{equation}
        \sigma_{S}=\sqrt{{\frac{{\displaystyle \sum_{i=1}^{N}\left(S_{i}-\bar{S}\right)^{2}}}{N}}}
        \wedge 
        \sigma_{t}=\sqrt{{\frac{{\displaystyle \sum_{i=1}^{N}\left(t_{i}-\bar{t}\right)^{2}}}{N}}}
    \end{equation}
    A partir de los calculos demostrados en la Tabla N°1 y N°2 se obtiene la sumatoria de la diferencia de cuadrados y se considera que N=20
    \begin{align*}
        &\sigma_{S}=\sqrt{{\frac{0,839}{20}}}
        &\wedge&
        &\sigma_{t}=\sqrt{{\frac{41,71}{20}}}\\
        &\sigma_{S}=0,2
        &\wedge&
        &\sigma_{t}=1,44 \tag{C}
    \end{align*}
\subsection{Desviación estandar del promedio o error típico del promedio de una variable} 
    \begin{equation}
        \sigma_{m}=\frac{\sigma}{\sqrt{N}}
    \end{equation}
    \begin{align*}
        &\sigma_{mS}=\frac{\sigma_{S}}{\sqrt{N}}
        &\wedge&
        &\sigma_{mt}=\frac{\sigma_{t}}{\sqrt{N}}\\
        &\sigma_{mS}=\frac{0,2}{\sqrt{20}}
        &\wedge&
        &\sigma_{mt}=\frac{1,44}{\sqrt{20}}\\
        &\sigma_{mS}=\frac{0,2}{4,47}
        &\wedge&
        &\sigma_{mt}=\frac{1,44}{4,47}\\
        &\sigma_{mS}=0,045
        &\wedge&
        &\sigma_{mt}=0,32 \tag{D}\\
    \end{align*}
\subsection{Error absoluto de la variable tiempo y distancia}
Se toma en cuenta que ambas medidas fueron cuantificadas de forma directa.\\
Se considera la ecuación del error absoluto de una magnitud física, ecuación 5.\\ 
    \begin{align*}
        &\Delta t=2\sigma_{mt}+EI_{t}
        &\wedge&
        &\Delta S=2\sigma_{mS}+EI_{S}
    \end{align*}
    Reemplazando los valores de E.I. obtenidos en el resultado A y B y la desviacion estandar del promedio opbtenidos en D
    \begin{align*}
        &\Delta t=2*0,32+0,01 s
        &\wedge&
        &\Delta S=2*0.045+0,0005 m\\
        &\Delta t=0,65 s
        &\wedge&
        &\Delta S=0,0905 m \tag{E}\\
    \end{align*}
\subsection{Estimacion del valor de la magintud de la distancia y tiempo}
Como se ha mencionado anteriormente, la variable distancia es una medida directa, por lo tanto, se toma el valor del error absoluto calculado en la resolución E y el valor del promedio obtenido a partir de los datos de la Tabla N°1.\\
Para el tiempo también se considera como una medida directa, por lo tanto, se toma el valor del error absoluto calculado en la resolución E y el valordel promedio obtenido a partir de los datos de la Tabla N°2.\\
    \begin{align*}
        &S=\bar{S}\pm \Delta S
        &\wedge&
        &t=\bar{t}\pm \Delta t\\
        &S=0,106\pm 0,0905 m
        &\wedge&
        &t=0,29\pm 0,65 s \tag{F}\\
    \end{align*}
\subsection{error relativo y porcentual de una variable}
    \begin{equation}
        E_{r}=\frac{\Delta x}{\bar{x}}
    \end{equation}
    \begin{equation}
        E_{r}=\frac{\Delta x}{\bar{x}}*100\%
    \end{equation}
    A partir de la ecuación anterior se estimó el error relativo y porcentual de cada vairable considerando que ambas son medidas directas. 
    \begin{align*}
        &E_{rs}=\frac{\Delta S}{\bar{S}}
        &\wedge&
        &E_{rt}=\frac{\Delta t}{\bar{t}}\\
        &E_{rs}=\frac{0,095}{0,106}
        &\wedge&
        &E_{rt}=\frac{0,65}{0,29}\\
        &E_{rs}=0,89
        &\wedge&
        &E_{rt}=2,26 \tag{G}\\
        &E_{rs\%}=89\% 
        &\wedge&
        &E_{rt\%}=226\% \tag{H}\\
    \end{align*}
\subsection{Ecuación de la gravedad}
    \begin{equation}
        S =-\frac{1}{2}gt^{2}
    \end{equation}
    Se considerara el valor de la gravedad como no negativa, por lo tanto se contempla que no es posición sino mas bien altura medida para que la ecuación tenga valores positivos\\
    Se contempla $\vec{g}=9,8 \frac{m}{{s}^{2}}$ y el valor del tiempo y la altura son el valor promedio de cada magnitud física \\
    \begin{align*}
        \bar{S} & =\frac{1}{2}\overrightarrow{g}\bar{t}^{2}\\
        \bar{S} & =\frac{1}{2}\overrightarrow{g}\bar{t}^{2} / *2\\
        2\bar{S}& =\overrightarrow{g}t^{2}/ *\frac{1}{\overrightarrow{g}}\\  
        \frac{2\bar{S}}{\overrightarrow{g}} & =\bar{t}^{2}
    \end{align*}
    \begin{equation}
        \bar{t}=\sqrt{\frac{2\bar{S}}{\overrightarrow{g}}}\\
    \end{equation}    
    A partir de la ecuación anterior (ecuación 12) se puede obtener el promedio de la variable tiempo como una medida indirecta
    Si el promedio de la variable distancia es 0,106 m y la gravedad es $\overrightarrow{g}=9,8 \frac{m}{s^{2}}$
    \begin{align*}
        \bar{t}&=\sqrt{\frac{2*0,106 m}{9,8\frac{m}{s^{2}}}}\\
        \bar{t}&=\sqrt{0,0216 s^{2}}\\
        \bar{t}&=0,15 s\tag{I}\\
    \end{align*}
    \subsection{Estimación del error en una medida indirecta}
    \begin{equation}
        \Delta F=2\sigma_{m}=2{\sqrt{{\displaystyle\sum_{i=1}^{N}(\sigma_{mx})^{2} (\frac{\partial F}{\partial x_{i}})^{2}}}}
    \end{equation}
    Para estimar el error absoluto del tiempo como una medida indirecta se debe calcular la derivada parcial y obtener el valor del error absoluto de la distancia estimada en la resolución F.
    \begin{equation}
        \Delta t=2\sigma_{mt}=2{\sqrt{(\Delta \bar{S})^{2} (\frac{\partial t}{\partial \bar{S}})^{2}}}
    \end{equation}
    Reemplazando los valores obtenidos del error absoluto de la distancia (Resolución F) y el valor de la derivada parcial del tiempo versus distancia (Resolución L), se resuelve:
    \begin{align*}
        \Delta t=2\sigma_{mt}=2{\sqrt{(0,095)^{2} (0,691)^{2}}}\\
        \Delta t=2\sigma_{mt}=2{\sqrt{(0,009025) (0,4774)}}\\
        \Delta t=2\sigma_{mt}=2{\sqrt{0,0656}}\\
        \Delta t=2\sigma_{mt}=2*0,06564\\
        \Delta t=\sigma_{mt}=0,13s\tag{J}\\
    \end{align*}
    \subsection{Magnitud de la variable tiempo como medida indirecta}
    Para estimar el valor de la magnitud del tiempo como medida indirecta, se estima el error absoluto por medio de la ecuación 14, obteniendose como resultado la resolución J y el valor del promedio se obtiene a partir de la
    ecuación 15, obteniendose como resultado la resolución M.
    \begin{align*}
        &t=\bar{t}\pm \Delta t\\
        &t=0,15 \pm0,13 s
    \end{align*} 
    \subsection{Estimación del error relativo y porcentual de la variable tiempo como medida indirecta}
    A partir de la ecuación 9 y 10 se estimó el error relativo y porcentual de la variable tiempo como medida indirecta. 
    \begin{align*}
        &E_{rt}=\frac{\Delta t}{\bar{t}}\\
        &E_{rt}=\frac{0,13}{0,25}\\
        &E_{rt}=0,52 \tag{K}\\
        &E_{rt\%}=52\% \tag{L}
    \end{align*}
    \subsection{Calculo de derivada parcial del tiempo en función de la distancia}
    Para obtener la derivada parcial $\frac{\partial t}{\partial S}$ se resuelve por medio de la ecuación 12.
    \begin{align*}
        \bar{t}=\sqrt{\frac{2}{\overrightarrow{g}}}*\sqrt{\bar{S}}\\
    \end{align*}
    Como la gravedad es una constante y la altura es la variable dependiente del tiempo, por lo tanto se deja sola la variable $\bar{S}$ para determinar la derivada parcial\\
    Se considera $\vec{g}=9,8 \frac{m}{{s}^{2}}$
    \begin{align*}
        \bar{t} & =\sqrt{\frac{2}{{9,8 \frac{m}{{s}^{2}}}}}*\sqrt{\bar{S}}\\
        \bar{t} & =\sqrt{0,2}*\sqrt{\bar{S}}\\ 
    \end{align*}
    \begin{equation}
        \bar{t} =0,45*\bar{S}^{\frac{1}{2}}\\
    \end{equation}
    \begin{align*}
        \bar{t} &=0,45\sqrt{\bar{S}}\\
        \bar{t} &=0,45\sqrt{0,106}\\
        \bar{t} &=0,45*0,3256\\
        \bar{t} &=0,147 s \tag{M}\\
    \end{align*}
    Una vez obtenida la ecuación 15 por el despeje de la variable dependiente. Se calcula la derivada parcial en función de la altura. 
    \begin{align*}
        \bar{t}                          & = 0,45*\bar{S}^{\frac{1}{2}} / \frac{\partial}{\partial S} \\
        \frac{\partial t}{\partial S}    & = 0,45*\bar{S}^{\frac{1}{2}-1}\\
        \frac{\partial t}{\partial S}    & = 0,45*\frac{1}{2}*\bar{S}^{-\frac{1}{2}}\\
        \frac{\partial t}{\partial S}    & = 0,225*\frac{1}{\bar{S}^{\frac{1}{2}}}\\
    \end{align*}
    \begin{equation}
        \frac{\partial t}{\partial S}      = \frac{0,225}{\sqrt{\bar{S}}}\\
    \end{equation}
    \begin{align*}
        \frac{\partial t}{\partial S} = \frac{0,225}{\sqrt{0,106}}\\
        \frac{\partial t}{\partial S} = 0,691 \tag{N}\\
    \end{align*}
    \section{Discusión}
    \section{Conclusiones}

    %$[2]$ González, J. $(2021)$. Muchos datos, distribución Gaussiana y análisis estadístico $[Apunte]$. Universidad Andrés Bello, Santiago,Chile.\\
    %$[3]$ Medina, Luis U., & Díaz, Sergio E. $(2018)$. Propagación de las incertidumbres en las mediciones aplicada a la identificación en el dominio de la frecuencia de matrices de inercia, rigidez y amortiguación de sistemas mecánicos. Ingeniería y Desarrollo, 36(1), 119-137. https://doi.org/10.14482/inde.36.1.1094\\
    
\end{document}

@article{Firestone_article,
author = "Firestone, C., & Scholl, B. J.",
title = "Cognition does not affect perception: Evaluating the evidence for “top-down” effects. Behavioral and brain sciences",
journal = "",
year = "2016",
number = "e229",
pages = "39",
volume = "e229",
note = "Notas opcionales"
}
